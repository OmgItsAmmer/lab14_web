\documentclass[12pt,a4paper]{article}
\usepackage[utf8]{inputenc}
\usepackage{geometry}
\usepackage{listings}
\usepackage{xcolor}
\usepackage{graphicx}
\usepackage{hyperref}
\usepackage{fancyhdr}
\usepackage{tcolorbox}

\geometry{margin=1in}

% Code styling
\definecolor{codebackground}{RGB}{26, 26, 46}
\definecolor{codepink}{RGB}{255, 105, 180}
\definecolor{codegreen}{RGB}{76, 175, 80}
\definecolor{codecomment}{RGB}{128, 128, 128}

\lstdefinestyle{reactstyle}{
    backgroundcolor=\color{codebackground},
    commentstyle=\color{codecomment},
    keywordstyle=\color{codepink},
    numberstyle=\tiny\color{codecomment},
    stringstyle=\color{codegreen},
    basicstyle=\ttfamily\footnotesize\color{white},
    breakatwhitespace=false,
    breaklines=true,
    captionpos=b,
    keepspaces=true,
    numbers=left,
    numbersep=5pt,
    showspaces=false,
    showstringspaces=false,
    showtabs=false,
    tabsize=2,
    frame=single,
    rulecolor=\color{codepink}
}

\lstset{style=reactstyle}

\pagestyle{fancy}
\fancyhf{}
\rhead{Lab 14 - Book Dashboard App}
\lhead{CS-344 Web Engineering}
\cfoot{\thepage}

\begin{document}

% Title Page
\begin{titlepage}
    \centering
    \vspace*{2cm}
    
    {\Huge\bfseries Department of Computing\par}
    \vspace{1cm}
    {\Large CS-344: Web Engineering\par}
    \vspace{1.5cm}
    
    {\huge\bfseries Lab 14: Book Dashboard App\par}
    \vspace{0.5cm}
    {\Large State Management \& Context in React\par}
    \vspace{2cm}
    
    \begin{tcolorbox}[colback=pink!10, colframe=pink!60, width=10cm]
        \centering
        \textbf{Class:} BESE-14AB \\
        \textbf{Date:} 18.12.25 \\
        \textbf{Instructor:} Ms. Naema Asif
    \end{tcolorbox}
    
    \vfill
    
    {\large Student Name: [Your Name]\par}
    {\large Registration No.: [Your Registration No.]\par}
    {\large Section: BESE-14AB\par}
    
    \vfill
    
    {\large \today\par}
\end{titlepage}

\tableofcontents
\newpage

\section{Introduction}

This lab explores advanced React concepts focusing on state management with objects and arrays, and implementing global state using the Context API. The project builds a comprehensive Book Dashboard application that demonstrates real-world state management patterns.

\subsection{Learning Objectives}
\begin{itemize}
    \item Manage objects and arrays in React state
    \item Update nested state immutably using the spread operator
    \item Create and use React Context for global state
    \item Build role-based views (Admin vs User)
    \item Implement CRUD operations
\end{itemize}

\section{Section 1: State with Objects}

\subsection{Concept Overview}
In React, when managing object state, we must update it immutably. This means creating a new object rather than modifying the existing one. The spread operator (\texttt{...}) helps achieve this.

\subsection{Profile Editor Component}

The Profile Editor demonstrates how to manage an object containing multiple properties (name and age) and update them independently.

\begin{lstlisting}[language=JavaScript, caption=ProfileEditor Component]
import React, { useState } from 'react';

function ProfileEditor() {
  // Initialize state with an object
  const [profile, setProfile] = useState({
    name: 'John Doe',
    age: 25
  });

  // Update name while preserving other properties
  const handleNameChange = (e) => {
    setProfile({
      ...profile,           // Copy existing properties
      name: e.target.value  // Update only name
    });
  };

  // Update age while preserving other properties
  const handleAgeChange = (e) => {
    setProfile({
      ...profile,          // Copy existing properties
      age: e.target.value  // Update only age
    });
  };

  return (
    <div>
      <h2>Profile Editor</h2>
      <input
        type="text"
        value={profile.name}
        onChange={handleNameChange}
      />
      <input
        type="number"
        value={profile.age}
        onChange={handleAgeChange}
      />
    </div>
  );
}
\end{lstlisting}

\subsection{Key Takeaways}
\begin{itemize}
    \item The spread operator creates a shallow copy of the object
    \item Only the specified property is updated
    \item React detects the new object reference and triggers a re-render
    \item This pattern prevents accidental property loss
\end{itemize}

\section{Section 2: Arrays of Objects}

\subsection{Concept Overview}
Managing arrays of objects requires careful handling to maintain immutability. Common operations include adding, removing, and updating items using array methods like \texttt{map()}, \texttt{filter()}, and the spread operator.

\subsection{Todo List Component}

The Todo List component demonstrates full CRUD operations on an array of todo objects.

\begin{lstlisting}[language=JavaScript, caption=Todo List - Adding Items]
function TodoList() {
  const [todos, setTodos] = useState([
    { id: 1, title: 'Learn React', completed: false }
  ]);
  
  const [newTodo, setNewTodo] = useState('');

  // Add new todo with unique ID
  const handleAddTodo = (e) => {
    e.preventDefault();
    
    const newTodoItem = {
      id: Date.now(),      // Generate unique ID
      title: newTodo,
      completed: false
    };
    
    // Create new array with existing and new item
    setTodos([...todos, newTodoItem]);
    setNewTodo('');        // Clear input
  };
  
  // Rest of component...
}
\end{lstlisting}

\begin{lstlisting}[language=JavaScript, caption=Todo List - Deleting Items]
// Delete todo by filtering out the item with matching ID
const handleDeleteTodo = (id) => {
  setTodos(todos.filter(todo => todo.id !== id));
};
\end{lstlisting}

\begin{lstlisting}[language=JavaScript, caption=Todo List - Updating Items]
// Toggle completed status
const handleToggleComplete = (id) => {
  setTodos(todos.map(todo => 
    todo.id === id 
      ? { ...todo, completed: !todo.completed }
      : todo
  ));
};
\end{lstlisting}

\subsection{Key Array Operations}
\begin{itemize}
    \item \textbf{Add:} Use spread operator to create new array with added item
    \item \textbf{Delete:} Use \texttt{filter()} to create array without the item
    \item \textbf{Update:} Use \texttt{map()} to create array with updated item
    \item \textbf{Unique IDs:} Use \texttt{Date.now()} for simple unique identifiers
\end{itemize}

\section{Section 3: Global State with Context}

\subsection{Concept Overview}
React Context solves the "prop drilling" problem by providing a way to share state across the entire component tree without passing props through every level.

\subsection{Role Context Implementation}

\begin{lstlisting}[language=JavaScript, caption=Creating Role Context]
import React, { createContext, useContext, useState } from 'react';

// Create context
const RoleContext = createContext();

// Custom hook for easy access
export const useRole = () => {
  const context = useContext(RoleContext);
  if (!context) {
    throw new Error('useRole must be used within RoleProvider');
  }
  return context;
};

// Provider component
export const RoleProvider = ({ children }) => {
  const [role, setRole] = useState('user');

  const toggleRole = () => {
    setRole(prevRole => 
      prevRole === 'user' ? 'admin' : 'user'
    );
  };

  const value = {
    role,
    toggleRole,
    isAdmin: role === 'admin',
    isUser: role === 'user'
  };

  return (
    <RoleContext.Provider value={value}>
      {children}
    </RoleContext.Provider>
  );
};
\end{lstlisting}

\subsection{Using Context in Components}

\begin{lstlisting}[language=JavaScript, caption=Role-Based Rendering]
function BookCard({ book, onEdit, onDelete }) {
  // Access context anywhere in the tree
  const { isAdmin } = useRole();

  return (
    <div className="card">
      <h3>{book.title}</h3>
      <p>{book.author}</p>
      
      {/* Conditional rendering based on role */}
      {isAdmin && (
        <div>
          <button onClick={() => onEdit(book)}>Edit</button>
          <button onClick={() => onDelete(book.id)}>Delete</button>
        </div>
      )}
    </div>
  );
}
\end{lstlisting}

\subsection{Context Benefits}
\begin{itemize}
    \item Eliminates prop drilling through multiple component levels
    \item Centralized state management
    \item Easy to add new consumers
    \item Clean component interfaces
\end{itemize}

\section{Final Project: Book Dashboard}

\subsection{Project Overview}
The Book Dashboard combines all learned concepts into a complete application with full CRUD functionality, role-based access control, and search capabilities.

\subsection{Main Features}

\subsubsection{Book Management}
\begin{itemize}
    \item Display books in a responsive grid
    \item Add new books (Admin only)
    \item Edit existing books (Admin only)
    \item Delete books with confirmation (Admin only)
    \item Search books by title or author
\end{itemize}

\subsubsection{Role-Based Access}
\begin{itemize}
    \item \textbf{Admin Role:} Full CRUD access with visible action buttons
    \item \textbf{User Role:} Read-only access, no action buttons shown
    \item Global role switcher accessible from any component
\end{itemize}

\subsection{Dashboard Implementation}

\begin{lstlisting}[language=JavaScript, caption=Book Dashboard Core Logic]
function BookDashboard() {
  const { isAdmin } = useRole();
  
  const [books, setBooks] = useState([
    { id: 1, title: 'The Great Gatsby', 
      author: 'F. Scott Fitzgerald', year: 1925 }
  ]);
  
  const [searchTerm, setSearchTerm] = useState('');

  // Add book
  const handleAddBook = (newBook) => {
    const bookWithId = {
      ...newBook,
      id: Date.now()
    };
    setBooks([...books, bookWithId]);
  };

  // Delete book
  const handleDeleteBook = (id) => {
    setBooks(books.filter(book => book.id !== id));
  };

  // Update book
  const handleSaveEdit = (updatedBook) => {
    setBooks(books.map(book => 
      book.id === updatedBook.id ? updatedBook : book
    ));
  };

  // Filter books
  const filteredBooks = books.filter(book => 
    book.title.toLowerCase().includes(searchTerm.toLowerCase()) ||
    book.author.toLowerCase().includes(searchTerm.toLowerCase())
  );

  return (
    <div>
      <input
        type="text"
        value={searchTerm}
        onChange={(e) => setSearchTerm(e.target.value)}
        placeholder="Search books..."
      />
      
      {isAdmin && (
        <button onClick={handleAddBook}>Add Book</button>
      )}
      
      {filteredBooks.map(book => (
        <BookCard
          key={book.id}
          book={book}
          onEdit={handleEditBook}
          onDelete={handleDeleteBook}
        />
      ))}
    </div>
  );
}
\end{lstlisting}

\section{Design Implementation}

\subsection{Color Scheme}
The application uses a dark theme with pink accents as specified:

\begin{itemize}
    \item \textbf{Primary Color:} Dark Light Pink (\#FF69B4)
    \item \textbf{Background:} Dark Navy (\#1a1a2e)
    \item \textbf{Surface:} Dark Blue (\#16213e)
    \item \textbf{Text:} White/Light Gray
\end{itemize}

\subsection{CSS Variables}

\begin{lstlisting}[language=CSS, caption=CSS Color Variables]
:root {
  --primary-color: #FF69B4;
  --primary-light: #FFB6D9;
  --primary-dark: #C71585;
  --background: #1a1a2e;
  --surface: #16213e;
  --text-primary: #ffffff;
  --border: #0f3460;
}

button {
  background-color: var(--primary-color);
  color: white;
  border: none;
  padding: 10px 20px;
  border-radius: 8px;
  transition: all 0.3s ease;
}

button:hover {
  background-color: var(--primary-dark);
  transform: translateY(-2px);
}
\end{lstlisting}

\section{Key Concepts Summary}

\subsection{Why Use the Spread Operator?}

The spread operator is essential for maintaining immutability in React:

\begin{enumerate}
    \item \textbf{Immutability:} Creates new references, allowing React to detect changes
    \item \textbf{Prevents Bugs:} Avoids accidental mutations of existing state
    \item \textbf{Performance:} Enables React's optimization through reference comparison
    \item \textbf{Predictability:} Makes state updates explicit and traceable
\end{enumerate}

\subsection{Problems Context Solves}

\begin{enumerate}
    \item \textbf{Prop Drilling:} Eliminates passing props through multiple levels
    \item \textbf{Code Duplication:} Centralized state reduces redundant code
    \item \textbf{Maintainability:} Changes to shared state in one place
    \item \textbf{Component Coupling:} Reduces dependencies between parent-child components
\end{enumerate}

\subsection{Benefits of Component Splitting}

\begin{enumerate}
    \item \textbf{Reusability:} Components can be used in multiple places
    \item \textbf{Testability:} Smaller components are easier to test
    \item \textbf{Readability:} Clear separation of concerns
    \item \textbf{Maintainability:} Changes isolated to specific components
\end{enumerate}

\section{Project Structure}

\begin{verbatim}
book-dashboard-app/
├── src/
│   ├── components/
│   │   ├── Section1/
│   │   │   └── ProfileEditor.jsx
│   │   ├── Section2/
│   │   │   └── TodoList.jsx
│   │   ├── Section3/
│   │   │   ├── RoleSwitcher.jsx
│   │   │   └── BookListBasic.jsx
│   │   └── BookDashboard/
│   │       ├── BookDashboard.jsx
│   │       ├── BookCard.jsx
│   │       ├── AddBookForm.jsx
│   │       └── EditBookModal.jsx
│   ├── contexts/
│   │   └── RoleContext.jsx
│   ├── App.jsx
│   ├── main.jsx
│   └── index.css
├── index.html
├── package.json
└── README.md
\end{verbatim}

\section{Running the Application}

\subsection{Installation}

\begin{verbatim}
# Install dependencies
npm install

# Run development server
npm run dev

# Build for production
npm run build
\end{verbatim}

\subsection{Features Demonstration}

\begin{enumerate}
    \item \textbf{Profile Editor:} Edit name and age to see object state updates
    \item \textbf{Todo List:} Add, edit, complete, and delete todos
    \item \textbf{Role Switcher:} Toggle between Admin and User modes
    \item \textbf{Book Dashboard:} 
    \begin{itemize}
        \item As Admin: Add, edit, and delete books
        \item As User: Browse books in read-only mode
        \item Search functionality works in both modes
    \end{itemize}
\end{enumerate}

\section{Conclusion}

This lab successfully demonstrates core React state management concepts:

\begin{itemize}
    \item Managing complex state with objects and arrays
    \item Immutable updates using the spread operator
    \item Global state management with Context API
    \item Role-based conditional rendering
    \item Complete CRUD operations
    \item Modern UI design with custom theming
\end{itemize}

The Book Dashboard combines these concepts into a practical, real-world application that showcases professional React development patterns.

\section{Learning Outcomes Achieved}

\begin{enumerate}
    \item Successfully implemented object state management with immutable updates
    \item Mastered array manipulation operations (add, delete, update)
    \item Created and consumed React Context for global state
    \item Built role-based access control system
    \item Developed a complete, functional book management application
    \item Applied modern CSS styling with custom color schemes
    \item Structured a scalable React application
\end{enumerate}

\vfill

\begin{center}
\textbf{End of Document}

\vspace{1cm}

CS-344 Web Engineering \\
Department of Computing \\
Lab 14: Book Dashboard App
\end{center}

\end{document}

